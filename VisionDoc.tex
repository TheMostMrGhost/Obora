\documentclass[10pt]{article}
\usepackage{polski}
\usepackage[utf8]{inputenc}
\usepackage{amssymb}
\usepackage{parskip}
\usepackage{amsthm}
\usepackage[dvipsnames]{xcolor}
\usepackage{tikz}
\usepackage{geometry}
\usepackage{hyperref}
\usepackage{makeidx}
\usepackage{amsmath}
\makeindex
\pagestyle{headings}

\date{}
\title{\vspace*{-30pt} Bazy danych, projekt zaliczniowy}
\author{Mikołaj Duch, Adam Wojciechowski}
\begin{document}

    \maketitle

    Projekt ma na celu obsługę gier na internetowego serwisu gier. 
    Wobec tego stworzona przez nas baza danych powinna umożliwiać:
    \begin{enumerate}
        \item Tworzenie kont użytkowników: amatorskie, profesjonalne, drużynowe.
        \item Dodawanie/usuwanie znajomych.
        \item Tworzenie rankingów, przy czym:
        \begin{enumerate}
            \item Domyślnym rankingiem dla każdej gry będzie ranking Elo. 
            \item Możliwa będzie także reprezentacja innego sposobu wyliczania rankingu, jak np. w brydżu 
            sportowym.
            \item Rankingi będzie można dostosowywać według pewnych kryteriów, takich jak ranking 
            potyczek przeciw konkretnemu graczowi, czy ranking w regionie.
            \item Tworzenie rankingów turniejowych, obowiazujących w ramach danego konkursu.
        \end{enumerate}
        \item Obsługę turniejów, w tym:
            \begin{enumerate}
                \item Wybieranie reprezentacji regionu na podstawie określonego rankingu.
                \item Tworzenie grup początkowych z uwzględnieniem siły graczy.
                \item Prezentacja wyników turnieju w formie drabinki.
            \end{enumerate}
        \item Zapis przebiegu partii.
        \item Pobranie przebiegu partii.
        \item Prezentacja statystyk zbiorczych: łączna liczba wygranych, przegranych, procentowa liczba wygranych do przegranych.
        \item Dodawanie nowych gier.
    \end{enumerate}

    Dane dotyczące rankingów i wyników pozyskiwać będziemy, o ile to możliwe, z danych rzeczywistych.
    Dla gier które nie mają powszechnie dostępnych list rankingowych będziemy pobierać dane z innego 
    rankingu (zapewne z turniejów sportowych), o ile prezentowane tam dane będą pasowały do kontekstu. 
    W razie gdyby takich nie było będziemy je wymyślać/modyfikować.

    


\end{document}